\documentclass[11pt]{article}
\usepackage[english]{babel}
\usepackage[letterpaper,top=2cm,bottom=2cm,left=3cm,right=3cm,marginparwidth=1.75cm]{geometry}
\usepackage{amsmath}
\usepackage{booktabs}
\usepackage{caption}
\usepackage{graphicx}
\usepackage{makecell}
\usepackage{pdflscape}
\usepackage{parskip}


\newcommand*{\PathToOutput}{../output/}


\begin{document}

\title{Evaporating Liquidity - Replication Report}

\author{
    Ruilong Guo,
    Sifei Zhao,
    Zhiyuan Liu,
    Junhan Fu
}


\maketitle

\section{Background}
Liquidity evaporated in many sectors of financial markets during the financial crisis 2007-09.
There are at least two possible explanations for this disappearance of market liquidity. One is that the crisis amplified asymmetric information problems.
An alternative and complementary theory is that the market turmoil strained the inventory-absorption capacity of the market- making sector, 
either because of a surge in liquidity demand from the public, or because market makers reduced 
liquidity supply in response to elevated levels of risk, tighter funding constraints, and reduced competition.

This paper studies this second channel using data from equity markets. 
The main objective is to estimate the extent by which the expected return from liquidity provision rises in times of financial market turmoil.

To construct a proxy for the returns from liquidity provision, Nagel examines reversal strategies that buy stocks that went down over the prior days,
 and sell stocks that went up during the prior days. The author shows that reversal strategy returns closely track the returns earned by liquidity providers.
 Moreover, the returns of the reversal strategy are highly predictable by the VIX index, which is a measure of the expected volatility of the S\&P 500 index.


\section{Methodology}
This project replicates Table 1 and 2 in Evaporating Liquidity\cite{nagel}. The author
shows that the returns of short-term reversal strategies are generated by liquidity 
provision, and therefore are highly predictable by the VIX index. The author also 
found that reversal strategies on not only individual stocks but also industry portfolios 
produce high returns, especially during periods of high VIX.

The author constructs the reversal strategy by averaging the returns of five substrategies
that weight stocks (or industries) proportional to the negative of market-adjusted returns
on days $t-1$ to $t-5$.
\begin{equation}
    w_{it}^R = -\left( \frac{1}{2} \sum_{i=1}^{N} \left| R_{it-1} - R_{mt-1} \right| \right)^{-1} \left( R_{it-1} - R_{mt-1} \right),
\end{equation}
where $R_{mt-1} = \frac{1}{N}\sum_{i=1}^N R_{it-1}$ is the equal-weighted market return.
Table 1 reports the summary statistics of the reversal strategies on individual stocks 
and industry portfolios. For individual stocks, the returns are calculated based on 
end-of-day transaction prices and quote midpoints.

Table 2 reports the results of the following predictive regression
\begin{equation}
    L_t^R = a + bVIX_{t-5} + c'g_{t-5} + e_t,
\end{equation}
where $L_t^R$ is the return of the reversal strategy. $VIX_{t-5}$ is the VIX index lagged
by 5 days, divided by $\sqrt{250}$. $g_{t-5}$ is a vector of control variables, including 
pre-decimalization dummy (takes a value of one prior to April 9, 2001 and a value of zero 
thereafter) and market return.

This project replicates these two tables using the same sample range as the original
paper (from January 1998 to December 2010). We also provide the updated tables using
data from January 1998 to December 2023.
\section{Data Description}
CRSP provides comprehensive historical data on stocks and indexes traded on U.S. exchanges, 
including NYSE, AMEX, and Nasdaq. In the project, CRSP daily stock data is pulled for 
constructing individual stock portfolios based on reversal strategy. Additionally, CRSP also 
provides daily index data, including indexes like CRSP value-weighted index, which is used 
to evaluate the performance of reversal strategy portfolios.

Fama/French Data Library database, created by Eugene Fama and Kenneth French, provides various 
financial data widely used in academic and investment research. The project pulls daily 
returns of 48 industry portfolios constructed based on Fama and French (1997) classification
from Fama/French Data Library database. The data is used to construct industry portfolios based
on reversal strategy.

FRED is a comprehensive source of economic data provided by the Federal Reserve Bank of St. Louis.
The project pulls the CBOE Volatility Index (VIX) data from FRED, which is used as a predictor 
variable in the predictive regression.


\section{Table Replication and Reproduction}
Table 1 reports the summary statistics of the reversal strategies on individual stocks based on 
end-of-day transaction prices and quote midpoints, and industry portfolios. Below three tables represents
the original table, the replicated table, and the updated table using data from January 1998 to December 2023.
Although the replication table can not be exactly the same as the original table, the results are consistent.

\begin{landscape}
    
\end{landscape}
\begin{table}
    \centering
    \caption*{Table 1:  Summary Statistics of Reversal Strategy Returns}

    \begin{tabular}{lccc}
    \toprule
    & Indiv. stock reversal & Indiv. stock reversal & Industry \\
    & Transact. prices & Quote-midpoints & Portfolio reversal \\
    \midrule
    \multicolumn{4}{c}{Panel A: Raw Returns} \\
    \midrule
    Mean return(\% per day) & 0.30 & 0.18 & 0.02 \\
    Std.dev.(\% per day) & 0.56 & 0.61 & 0.52 \\
    Skewness & 3.02 & 2.74 & 1.06 \\
    Kurtosis & 38.21 & 40.50 & 17.93 \\
    Worst day return(\%) & -3.88 & -4.76 & -3.93 \\
    Worst 3-month return(\%) & 2.56 & -2.13 & -9.28 \\
    Beta & 0.11 & 0.11 & 0.09 \\
    Annualized Sharpe Ratio & 8.44 & 4.50 & 0.56 \\
    \midrule
    \multicolumn{4}{c}{Panel B: Returns hedged for conditional market factor exposure} \\
    \midrule
    Mean return(\% per day) & 0.29 & 0.17 & 0.01 \\
    Std.dev.(\% per day) & 0.48 & 0.54 & 0.47 \\
    Skewness & 2.45 & 2.26 & 0.88 \\
    Kurtosis & 31.26 & 34.51 & 15.97 \\
    Worst day return(\%) & -2.26 & -3.92 & -3.12 \\
    Worst 3-month return(\%) & 2.27 & -1.28 & -7.97 \\
    Beta & 0.00 & 0.00 & 0.00 \\
    Annualized Sharpe Ratio & 9.58 & 4.91 & 0.44 \\
    \bottomrule
    \end{tabular}
\end{table}


\begin{table}
    \centering
    \caption*{Table 1:  Summary Statistics of Reversal Strategy Returns (Replicated)}

    \begin{tabular}{lccc}
\toprule
& Indiv. stock reversal & Indiv. stock reversal & Industry \\
& Transact. prices & Quote-midpoints & Portfolio reversal \\
\midrule
\multicolumn{4}{c}{Panel A: Raw Returns} \\
\midrule
Mean return(\% per day) & 0.31 & 0.19 & 0.02 \\
Std.dev.(\% per day) & 0.56 & 0.67 & 0.56 \\
Skewness & 3.01 & 3.58 & 0.77 \\
Kurtosis & 38.46 & 50.26 & 14.60 \\
Worst day return(\%) & -3.84 & -4.54 & -3.70 \\
Worst 3-month return(\%) & 2.51 & -2.72 & -12.17 \\
Beta & 0.11 & 0.09 & 0.10 \\
Annualized Sharpe Ratio & 8.61 & 4.54 & 0.45 \\
\midrule
\multicolumn{4}{c}{Panel B: Returns hedged for conditional market factor exposure} \\
\midrule
Mean return(\% per day) & 0.30 & 0.19 & 0.01 \\
Std.dev.(\% per day) & 0.54 & 0.65 & 0.54 \\
Skewness & 3.02 & 3.84 & 0.65 \\
Kurtosis & 39.00 & 55.98 & 12.20 \\
Worst day return(\%) & -3.05 & -3.96 & -3.31 \\
Worst 3-month return(\%) & 2.07 & -2.02 & -9.18 \\
Beta & 0.00 & 0.00 & 0.00 \\
Annualized Sharpe Ratio & 8.87 & 4.58 & 0.38 \\
\bottomrule
\end{tabular}
\end{table}


\begin{table}
    \centering
    \caption*{Table 1:  Summary Statistics of Reversal Strategy Returns (Updated)}

    \begin{tabular}{lccc}
\toprule
& Indiv. stock reversal & Indiv. stock reversal & Industry \\
& Transact. prices & Quote-midpoints & Portfolio reversal \\
\midrule
\multicolumn{4}{c}{Panel A: Raw Returns} \\
\midrule
Mean return(\% per day) & 0.23 & 0.16 & 0.01 \\
Std.dev.(\% per day) & 0.67 & 0.77 & 0.52 \\
Skewness & -0.51 & 4.97 & 0.70 \\
Kurtosis & 48.95 & 136.49 & 14.54 \\
Worst day return(\%) & -12.44 & -7.50 & -3.70 \\
Worst 3-month return(\%) & -7.53 & -9.62 & -12.17 \\
Beta & 0.12 & 0.10 & 0.09 \\
Annualized Sharpe Ratio & 5.39 & 3.29 & 0.32 \\
\midrule
\multicolumn{4}{c}{Panel B: Returns hedged for conditional market factor exposure} \\
\midrule
Mean return(\% per day) & 0.22 & 0.16 & 0.01 \\
Std.dev.(\% per day) & 0.65 & 0.76 & 0.50 \\
Skewness & -0.72 & 5.39 & 0.64 \\
Kurtosis & 52.57 & 151.89 & 12.62 \\
Worst day return(\%) & -12.47 & -7.49 & -3.30 \\
Worst 3-month return(\%) & -5.39 & -9.79 & -10.05 \\
Beta & -0.00 & -0.00 & -0.00 \\
Annualized Sharpe Ratio & 5.44 & 3.27 & 0.22 \\
\bottomrule
\end{tabular}
\end{table}
\end{landscape}

Table 2 reports the results of the predictive regression of reversal strategy returns on the VIX index, as well as 
other control variables. Below three tables represents the original table, the replicated table, and the updated table.
The replicated table has been verified that coefficients of relicated result are within the 99.7\% confidence interval of the original result.

\begin{landscape}
    \begin{table}[h!]
        \centering
        \caption*{Table 2: Predicting Reversal Strategy Returns with VIX}
        
        \small Original Table 2 from the paper.
        \medskip
        
        \scriptsize
        \begin{tabular}{lcccccccccccc}
\toprule
& \multicolumn{4}{c}{\makecell{Individual stocks\\Transaction-price returns}} & \multicolumn{4}{c}{\makecell{Individual stocks\\Quote-midpoint returns}} & \multicolumn{4}{c}{\makecell{Industry\\portfolios}} \\
& \multicolumn{3}{c}{Daily} & Monthly & \multicolumn{3}{c}{Daily} & Monthly & \multicolumn{3}{c}{Daily} & Monthly \\
& (1) & (2) & (3) & (4) & (5) & (6) & (7) & (8) & (9) & (10) & (11) & (12) \\
\midrule
Intercept & -0.03 & -0.05 & -0.02 & 0.02 & -0.06 & -0.07 & -0.04 & -0.01 & -0.08 & -0.09 & -0.06 & -0.05 \\
& (0.03) & (0.02) & (0.02) & (0.02) & (0.03) & (0.03) & (0.03) & (0.02) & (0.02) & (0.02) & (0.02) & (0.01) \\
VIX & 0.22 & 0.20 & 0.18 & 0.15 & 0.16 & 0.16 & 0.13 & 0.10 & 0.07 & 0.07 & 0.05 & 0.04 \\
& (0.02) & (0.02) & (0.02) & (0.01) & (0.02) & (0.02) & (0.02) & (0.02) & (0.02) & (0.02) & (0.02) & (0.01) \\
Pre-decim. & & 0.22 & 0.22 & 0.23 & & 0.08 & 0.09 & 0.09 & & 0.00 & 0.01 & 0.01 \\
& & (0.03) & (0.03) & (0.03) & & (0.03) & (0.03) & (0.03) & & (0.02) & (0.02) & (0.02) \\
$R_M$ & & & -0.60 & -0.03 & & & -0.59 & -0.16 & & & -0.42 & -0.05 \\
& & & (0.19) & (0.26) & & & (0.21) & (0.28) & & & (0.17) & (0.16) \\
Adj. $R^2$ & 0.07 & 0.11 & 0.11 & 0.56 & 0.03 & 0.03 & 0.04 & 0.25 & 0.01 & 0.01 & 0.01 & 0.07 \\
\bottomrule
\end{tabular}

    \end{table}
    
    
    \begin{table}[h!]
        \centering
        \caption*{Table 2: Predicting Reversal Strategy Returns with VIX (Replicated)}
    
        \raggedright
        \small Replicated Table 2, which uses the same sample range as the original (from 
        January 1998 to December 2010). 
        It has been verified that coefficients of predictor variables in the replicated result
        have the same sign with the original result. The coefficients of replicated result are
        within the 99.7\% confidence interval of the original result.
        \medskip
        
        \scriptsize
        \centering
        \begin{tabular}{lllllllllllll}
\toprule
 & \multicolumn{4}{r}{Individual stocks
Transaction-price returns} & \multicolumn{4}{r}{Individual stocks
Quote-midpoint returns} & \multicolumn{4}{r}{Industry
portfolios} \\
 & \multicolumn{3}{r}{Daily} & Monthly & \multicolumn{3}{r}{Daily} & Monthly & \multicolumn{3}{r}{Daily} & Monthly \\
 & (1) & (2) & (3) & (4) & (5) & (6) & (7) & (8) & (9) & (10) & (11) & (12) \\
\midrule
Intercept & -0.06 & -0.09 & -0.06 & -0.01 & -0.06 & -0.08 & -0.03 & 0.00 & -0.10 & -0.10 & -0.07 & -0.04 \\
 & (0.03) & (0.02) & (0.03) & (0.02) & (0.03) & (0.03) & (0.04) & (0.03) & (0.03) & (0.03) & (0.03) & (0.02) \\
VIX & 0.25 & 0.23 & 0.21 & 0.18 & 0.18 & 0.17 & 0.14 & 0.11 & 0.08 & 0.08 & 0.06 & 0.04 \\
 & (0.02) & (0.02) & (0.02) & (0.01) & (0.03) & (0.03) & (0.03) & (0.02) & (0.02) & (0.02) & (0.02) & (0.01) \\
Pre-decim. &  & 0.24 & 0.24 & 0.25 &  & 0.11 & 0.11 & 0.12 &  & 0.01 & 0.01 & 0.02 \\
 &  & (0.03) & (0.03) & (0.03) &  & (0.03) & (0.03) & (0.03) &  & (0.02) & (0.02) & (0.02) \\
$R_M$ &  &  & -0.45 & 0.10 &  &  & -0.78 & -0.27 &  &  & -0.57 & -0.21 \\
 &  &  & (0.19) & (0.23) &  &  & (0.23) & (0.26) &  &  & (0.21) & (0.16) \\
Adj. $R^2$ & 0.07 & 0.10 & 0.10 & 0.65 & 0.02 & 0.03 & 0.03 & 0.27 & 0.01 & 0.01 & 0.01 & 0.07 \\
\bottomrule
\end{tabular}

    \end{table}
    
    \begin{table}[h!]
        \centering
        \caption*{Table 2: Predicting Reversal Strategy Returns with VIX (Updated)}
    
        \small Updated Table 2, using data from January 1998 to December 2023.
        The results are consistent.
        \medskip
        
        \scriptsize
        \begin{tabular}{lcccccccccccc}
\toprule
 & \multicolumn{4}{c}{\makecell{Individual stocks\\Transaction-price returns}} & \multicolumn{4}{c}{\makecell{Individual stocks\\Quote-midpoint returns}} & \multicolumn{4}{c}{\makecell{Industry\\portfolios}} \\
 & \multicolumn{3}{c}{Daily} & Monthly & \multicolumn{3}{c}{Daily} & Monthly & \multicolumn{3}{c}{Daily} & Monthly \\
 & (1) & (2) & (3) & (4) & (5) & (6) & (7) & (8) & (9) & (10) & (11) & (12) \\
\midrule
Intercept & -0.08 & -0.08 & -0.05 & -0.01 & -0.09 & -0.09 & -0.06 & -0.02 & -0.09 & -0.09 & -0.07 & -0.06 \\
 & (0.02) & (0.03) & (0.02) & (0.02) & (0.03) & (0.03) & (0.03) & (0.03) & (0.02) & (0.02) & (0.02) & (0.02) \\
VIX & 0.24 & 0.21 & 0.19 & 0.15 & 0.19 & 0.18 & 0.17 & 0.12 & 0.08 & 0.08 & 0.07 & 0.05 \\
 & (0.02) & (0.02) & (0.02) & (0.02) & (0.02) & (0.03) & (0.02) & (0.03) & (0.02) & (0.02) & (0.02) & (0.01) \\
Pre-decim. &  & 0.26 & 0.27 & 0.28 &  & 0.09 & 0.10 & 0.12 &  & 0.00 & 0.00 & 0.01 \\
 &  & (0.03) & (0.03) & (0.03) &  & (0.03) & (0.03) & (0.03) &  & (0.02) & (0.02) & (0.02) \\
$R_M$ &  &  & -0.39 & 0.03 &  &  & -0.47 & -0.04 &  &  & -0.24 & -0.03 \\
 &  &  & (0.17) & (0.18) &  &  & (0.23) & (0.26) &  &  & (0.16) & (0.13) \\
Adj. $R^2$ & 0.04 & 0.05 & 0.05 & 0.53 & 0.02 & 0.02 & 0.02 & 0.19 & 0.01 & 0.01 & 0.01 & 0.08 \\
\bottomrule
\end{tabular}

    \end{table}
    
    \end{landscape}


\section{Additional Analysis}

\begin{landscape}
    \begin{table}
        \centering
        \caption*{Table 3: Additional Summary Statistics of Reversal Strategy Returns}
    
        \raggedright
        
        \small 
        Apart from the original statistical analysis of reversal strategy provided by 
        the paper, we create a new version of performance matrix which includes VaR(0.05), 
        CVaR(0.05), max drawdown, and other drawdown-based strategy perfomance, and we also 
        add CRSP value weighted index as the benchmark to evaluate the performance of reversal strategies. 
    
        Compared to the CRSP value weighted index, the reversal strategy based on individual 
        stocks tends to have much higher annualized mean return and lower annualized volatility, 
        which cause a way higher annualized sharpe ratio. The mean return of industry reversal 
        strategy is a little bit lower than the banchmark, but it has lower volatility 
        with higher sharpe ratio.
    
        With regard to max drawdown, the transact price based individual reversal strategy 
        is the best(-4.38\%) among all the reversal strategies(quote-midpoints: -7.70\%, 
        industry: -13.90\%) and the CRSP index(-57.18\%). That strategy dropped form the 
        peak on 2009-10-22 after the period of financial crisis. And it only used 7 days to 
        recover the lose since the peak, while the industry reversal strategy took 433 days 
        to recover and CRSP value weighted index didn't recover to the peak.
        \medskip
    
        \centering
        \begin{tabular}{lllll}
\toprule
 & Transact. prices & Quote-midpoints & Industry portfolio & CRSP Value Weighted Index \\
\midrule
Annualized Mean Return(%) & 76.97 & 47.94 & 4.02 & 7.86 \\
Annualzied Volatility(%) & 8.94 & 10.57 & 8.85 & 19.65 \\
Annualized Sharpe Ratio & 8.61 & 4.53 & 0.45 & 0.40 \\
Skewness & 3.01 & 3.57 & 0.77 & -0.27 \\
Kurtosis & 38.46 & 50.15 & 14.60 & 12.01 \\
VaR (0.05)(%) & -0.33 & -0.61 & -0.74 & -1.92 \\
CVaR (0.05)(%) & -0.67 & -1.02 & -1.22 & -2.96 \\
Max Drawdown(%) & -4.38 & -7.70 & -13.90 & -57.18 \\
Peak & 2000-04-11 & 2001-07-13 & 1998-04-09 & 2007-10-09 \\
Bottom & 2000-04-14 & 2001-09-21 & 1998-10-08 & 2009-03-09 \\
Recovery Date & 2000-04-18 & 2001-10-24 & 1999-06-16 & 2013-03-08 \\
Duration (days) & 7 & 103 & 433 & 1977 \\
\bottomrule
\end{tabular}

    \end{table}
    
    
    
    \begin{figure}
        \centering
        \includegraphics[width=0.8\textwidth]{\PathToOutput/reversal_strategy_vix.png}
        \caption{Reversal Strategy and VIX}
    
        \raggedright
        
        \small 
        This figure shows the three-month moving average return of the reversal strategy 
        and VIX index across 1998 to 2010. The blue curve(VIX index) has a pre-trend of 
        the red curve(3-month MA return of reversal strategy), which presents a key finding 
        of the paper that the VIX index has a power to predict the reversal strategy return.
        During the LTCM crisis in 1998 and  Nasdaq decline in 2000, the reversal strategy 
        return increased with VIX increasing. From then until 2007, returns declined steadily
        to less than 0.2\% per day, but during the financial crisis, they surged, surpassing 
        levels seen during the LTCM crisis. The figure illustrates a strong correlation 
        between the time variation in the reversal strategy's return and the VIX index. 
        Since the financial crisis began in 2007, the returns of the reversal strategy 
        and the VIX have closely tracked each other.
        \medskip
    \end{figure}
\end{landscape}





\section{Success and Challenges}

The main challenge of the project is the data collection and processing. The original paper uses 
CRSP daily stock data and Fama/French Data Library database to construct individual stock and 
industry portfolios based on reversal strategy. The project needs to pull the data from these two 
databases and process the data to construct the portfolios. The project also needs to pull the 
VIX data from FRED to use as a predictor variable in the predictive regression. The data collection 
and processing are time consuming and require careful attention to details.

While using GitHub to manage the project, the challenges are that the team members need to be pay 
attention to version control and the project needs to be well-organized to avoid conflicts. The
project also needs to be well-documented and well-tested to ensure the robustness of the results.

Since the team members can easily track the changes and collaborate well on the project, 
the project is well-organized and the team members are able to work efficiently. The project 
successfully overcomes the challenges and completes the replication. The results are consistent 
with the original paper, which further verifies the robustness of the original paper. The project 
also provides the updated tables using data from January 1998 to December 2023, which provides 
more recent evidence on the predictability of reversal strategy returns by the VIX index.


\section{Conclusion}
In summary, our endeavor to duplicate Tables 1 and 2 from "Evaporating Liquidity" encountered a mix of triumphs and hurdles, 
offering priceless insights. Our efforts to replicate yielded results that closely aligned with the originals, 
validating Stefan Nagel's rigorous approaches. Although we faced some difficulties, 
our project illustrates that through careful methodology and dedicated work, essential scholarly conclusions can be confirmed.


\newpage
\bibliographystyle{jpe}
\bibliography{bibliography.bib}

\end{document}